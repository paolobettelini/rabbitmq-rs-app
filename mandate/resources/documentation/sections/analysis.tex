\documentclass[../documentation.tex]{subfiles}

\begin{document}

\subsection{Requirements}

\bgroup{}
\def\arraystretch{1.25}
\begin{center}
    \begin{tabular}{ |l|p{9cm}| }
        \hline
        \multicolumn{2}{|c|}{\textbf{Req-00}} \\
        \hline
        \textbf{Name} & Login \& Register \\
        \hline
        \textbf{Priority} & 1 \\
        \hline
        \textbf{Version} & 1.0 \\
        \hline
        \textbf{Notes} & none \\
        \hline
        \textbf{Description} & The user must be able to create an account and log in. \\
        \hline
        \multicolumn{2}{|c|}{\textbf{Subrequirements}} \\
        \hline
        \textbf{Req-00\_0} & The authentication must be kept alive by a cookie. \\
        \textbf{Req-00\_1} & The keep-alive cookie must contain a randomly generated token. \\
        \textbf{Req-00\_2} & The password must be hashed client-side. \\
        \hline
    \end{tabular}
\end{center}
\egroup{}

\bgroup{}
\def\arraystretch{1.25}
\begin{center}
    \begin{tabular}{ |l|p{9cm}| }
        \hline
        \multicolumn{2}{|c|}{\textbf{Req-01}} \\
        \hline
        \textbf{Name} & Functionality \\
        \hline
        \textbf{Priority} & 1 \\
        \hline
        \textbf{Version} & 1.0 \\
        \hline
        \textbf{Notes} & none \\
        \hline
        \textbf{Description} & The website must contain a file dropzone. The user must be able to upload an image which will be converted into a 200x200 px webp. \\
        \hline
        \multicolumn{2}{|c|}{\textbf{Subrequirements}} \\
        \hline
        \textbf{Req-01\_0} & During the conversion an async progress status must be displayed. \\
        \hline
    \end{tabular}
\end{center}
\egroup{}

\bgroup{}
\def\arraystretch{1.25}
\begin{center}
    \begin{tabular}{ |l|p{9cm}| }
        \hline
        \multicolumn{2}{|c|}{\textbf{Req-02}} \\
        \hline
        \textbf{Name} & Message Queues \\
        \hline
        \textbf{Priority} & 1 \\
        \hline
        \textbf{Version} & 1.0 \\
        \hline
        \textbf{Notes} & none \\
        \hline
        \textbf{Description} & Every message sent between webserver and backend
            must be through a message queue on the message broker. \\
        \hline
    \end{tabular}
\end{center}
\egroup{}

\bgroup{}
\def\arraystretch{1.25}
\begin{center}
    \begin{tabular}{ |l|p{9cm}| }
        \hline
        \multicolumn{2}{|c|}{\textbf{Req-03}} \\
        \hline
        \textbf{Name} & Gallery \\
        \hline
        \textbf{Priority} & 1 \\
        \hline
        \textbf{Version} & 1.0 \\
        \hline
        \textbf{Notes} & none \\
        \hline
        \textbf{Description} & When the users logs in a list of the previously converted images must be display. \\
        \hline
        \multicolumn{2}{|c|}{\textbf{Subrequirements}} \\
        \hline
        \textbf{Req-03\_0} & Only the last \(N\) images are loaded. Another chunk of images is loaded if requested by the user. \\
        \hline
    \end{tabular}
\end{center}
\egroup{}

\bgroup{}
\def\arraystretch{1.25}
\begin{center}
    \begin{tabular}{ |l|p{9cm}| }
        \hline
        \multicolumn{2}{|c|}{\textbf{Req-04}} \\
        \hline
        \textbf{Name} & Network Structure \\
        \hline
        \textbf{Priority} & 1 \\
        \hline
        \textbf{Version} & 1.1 \\
        \hline
        \textbf{Notes} & none \\
        \hline
        \textbf{Description} & A loadbalancer (Round Robin) is the entry point for \(N\) webservers.
        There are \(M\) backend workers. Workers and webservers communicate by connecting to a RabbitMQ
        server or a RabbitMQ Cluster.
        Each worker stores data on the same database. \\
        \hline
    \end{tabular}
\end{center}
\egroup{}

\bgroup{}
\def\arraystretch{1.25}
\begin{center}
    \begin{tabular}{ |l|p{9cm}| }
        \hline
        \multicolumn{2}{|c|}{\textbf{Req-05}} \\
        \hline
        \textbf{Name} & Scalability \\
        \hline
        \textbf{Priority} & 1 \\
        \hline
        \textbf{Version} & 1.0 \\
        \hline
        \textbf{Notes} & none \\
        \hline
        \textbf{Description} & The network must scale horizontally with multiple servers. \\
        \hline
    \end{tabular}
\end{center}
\egroup{}

\pagebreak

\subsection{Use Cases}

The user can log in only if it has registered. Once logged the user has access
to the application features:
\begin{itemize}
    \item \textbf{Logout} (Logout)
    \item \textbf{Upload} (Upload an image)
    \item \textbf{Image} (Retrieve an image)
\end{itemize}

\begin{tikzpicture}
    \begin{umlsystem}[x=6] {} % empty title
        \umlusecase[x=1, name=login, width=2cm] {Login}
        \umlusecase[x=1, y=-5, name=register, width=2cm] {Register}
        \umlusecase[x=8, name=logout, width=2cm] {Logout}
        \umlusecase[x=8, y=-2,name=upload, width=2cm] {Upload}
        \umlusecase[x=8, y=-5, name=image, width=2cm] {Image}
    \end{umlsystem}
    \node [above] at (current bounding box.north) {Application};

    \umlactor[y=-2] {Actor}

    \umlassoc{Actor}{login}
    \umlassoc{Actor}{register}

    \draw [tikzuml dependency style] (login) -- node[above] {\(\ll extends \gg\)} (logout);
    \draw [tikzuml dependency style] (login) -- node[above] {\(\ll extends \gg\)} (upload);
    \draw [tikzuml dependency style] (login) -- node[above] {\(\ll extends \gg\)} (image);

    \draw [tikzuml dependency style] (login) -- node[left] {\(\ll includes \gg\)} (register);
\end{tikzpicture}

\end{document}