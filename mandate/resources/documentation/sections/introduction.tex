\documentclass[../documentation.tex]{subfiles}

\begin{document}

\subsection{Abstract}

The goal of this project is to make a network infrastructure
which extensively uses a messaging queue system (RabbitMQ).
My additional personal requirement is to use the Rust programming language
as much as possible.

\subsection{Information}

This is a project of the Scuola Arti e Mestieri di Trevano (SAMT) under the following circumstances

\begin{itemize}
    \item \textbf{Section}: Computer Science
    \item \textbf{Year:} Fourth
    \item \textbf{Class:} Progetti Individuali
    \item \textbf{Supervisor:} Geo Petrini
    \item \textbf{Title:} RabbitMQ prototype
    \item \textbf{Start date}: 2022-09-29
    \item \textbf{Deadline}: 2022-12-07
\end{itemize}

and the following requirements

\begin{itemize}
    \item \textbf{Documentation}: a full documentation of the work done
    \item \textbf{Diary}: constant changelog for each work session
    \item \textbf{Source code}: working source code of the project
\end{itemize}

All the source code and documents can be found at
\href{http://gitsam.cpt.local/2022\_2023\_1\_semestre/prototipo-rabbitmq}
{http://gitsam.cpt.local/2022\_2023\_1\_semestre/prototipo-rabbitmq}
\cite{gitrepo}.

% todo change to GitHub

\subsection{Structure}

This document is structured as such:

% TODO
\begin{enumerate}
    \item \hyperlink{section.1}{\textbf{Introduction:}}
        General information, requirements and scope of the project

    \item \hyperlink{section.2}{\textbf{Analysis:}}
        Analysis
\end{enumerate}

\end{document}