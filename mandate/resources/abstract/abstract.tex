\documentclass{article}

\usepackage{amsmath}
\usepackage{amssymb}
\usepackage{parskip}
\usepackage{fullpage}
\usepackage{hyperref}

\hypersetup{
    colorlinks=true,
    linkcolor=black,
    urlcolor=blue,
    pdftitle={Paolo Bettelini - Abstract},
    pdfpagemode=FullScreen,
}

\title{%
    RabbitMQ Infrastructure Prototype \\
    \large Abstract
}

\author{%
    Paolo Bettelini \\
    \large Scuola d'Arti e Mestieri di Trevano (SAMT)}

\date{}

\begin{document}

\maketitle

\vspace{2cm}

\textbf{Section}: Computer Science \\
\textbf{Year:} Fourth \\
\textbf{Class:} Progetti Individuali \\
\textbf{Supervisor:} Geo Petrini \\
\textbf{Title:} RabbitMQ based web app prototype \\
\textbf{Timeline}: 2022-09-29 - 2022-12-07

\vspace{2cm}

\thispagestyle{empty} % no page number

\section*{Abstract}

Message brokers have always been a critical part of many infrastructures.
This architectural pattern allows for easy to horizontally scale and reliable networks.
The goal of this project is to make a network infrastructure
which uses a messaging system through a message broker (RabbitMQ).
My additional personal requirement is to use the Rust programming language
as much as possible and bleeding edge technology.

\section*{Execution}

This project has been built using bleeding edge technology.
The Rust programming language is used extensively. Every software
developed, including the website, is based on a Rust codebase.
The creation of the network topology is eased by the use of Vagrant.
MariaDB and NginX are also used.

\section*{Results}

The goal of the project was reached and the requirements were met.
The project gave me the opportunity to gain a more deep
understanding of the Rust programming language. I am kind of unhappy
with how the project itself turned out; there are lots of things
that I could have done better. I hated that fact that
the infrastructure is, by design, inherently suboptimal,
though still desirable to learn about message brokers and such designs.
The application itself is not especially exciting
but rather kind of boring and trivial, but I managed
to make it more interesting by using the technologies that I chose.

\end{document}